%Example of use of oxmathproblems latex class for problem sheets
\documentclass{oxmathproblems}

%(un)comment this line to enable/disable output of any solutions in the file
%\printanswers

%define the page header/title info
\course{Algorithm Design and Analysis}
\sheettitle{Assignment 6 \\ Deadline: Jun 20, 2024} %can leave out if no title per sheet


\begin{document}
Choose \textbf{two} questions to solve.

\begin{questions}
\miquestion[50] 
Given an undirected graph $G=(V,E)$ and an integer $k$, decide if $G$ has a spanning tree with maximum degree at most $k$. Prove that this problem is NP-complete.

Solution:

To prove that deciding if a given undirected graph \( G = (V, E) \) has a spanning tree with maximum degree at most \( k \) is NP-complete, we need to show two things:
1. The problem is in NP.
2. The problem is NP-hard.

1. The Problem is in NP

To show that the problem is in NP, we need to demonstrate that given a solution, we can verify it in polynomial time. For this problem, a solution is a spanning tree \( T \) of \( G \) with maximum degree at most \( k \).

Given a candidate spanning tree \( T \):
\begin{enumerate}
    \item Check if \( T \) is a tree (connected and acyclic):
    \begin{itemize}
        \item We can verify if \( T \) is acyclic by ensuring there are no cycles. This can be done using depth-first search (DFS) or breadth-first search (BFS) in \( O(|V| + |E|) \) time.
        \item We can check if \( T \) is connected by performing a DFS or BFS from any vertex and ensuring all vertices are visited in \( O(|V| + |E|) \) time.
    \end{itemize}
    \item Check if \( T \) spans all vertices, i.e., \( T \) contains exactly \( |V| - 1 \) edges.
    \item Check if the maximum degree of \( T \) is at most \( k \). This can be done by iterating over all vertices and counting their degrees in \( O(|V|) \) time.
\end{enumerate}

Since all these checks can be done in polynomial time, the problem is in NP.

2. The Problem is NP-Hard

To show that the problem is NP-hard, we will use a reduction from a known NP-complete problem. A suitable choice is the \textit{Hamiltonian Path Problem}, which is known to be NP-complete.

Hamiltonian Path Problem:

\textbf{Instance:} A graph \( G' = (V', E') \).

\textbf{Question:} Does there exist a path in \( G' \) that visits each vertex exactly once?

Reduction from Hamiltonian Path to Our Problem:

Given an instance \( G' = (V', E') \) of the Hamiltonian Path Problem, we construct an instance \( G = (V, E) \) and an integer \( k \) for our problem as follows:

\begin{enumerate}
    \item Construct a new graph \( G \) by adding a new vertex \( v_0 \) to \( G' \) and connecting \( v_0 \) to every vertex in \( G' \). Formally, \( V = V' \cup \{v_0\} \) and \( E = E' \cup \{(v_0, v) \mid v \in V'\} \).
    \item Set \( k = 2 \).
\end{enumerate}

\textbf{Proof of Equivalence:}

\begin{itemize}
    \item \textit{(If Hamiltonian Path exists)} Suppose there exists a Hamiltonian path in \( G' \). We can construct a spanning tree \( T \) in \( G \) by:
    \begin{itemize}
        \item Adding the Hamiltonian path edges.
        \item Connecting the vertex \( v_0 \) to one of the endpoints of the Hamiltonian path.
    \end{itemize}
    In this spanning tree \( T \), \( v_0 \) has degree 1, the endpoints of the Hamiltonian path in \( G' \) have degree 2, and all other vertices in \( G' \) have degree 2 or 1 (depending on their position in the path). Therefore, the maximum degree in this spanning tree is at most 2, satisfying \( k = 2 \).

    \item \textit{(If spanning tree with maximum degree at most \( k \) exists)} Suppose there exists a spanning tree \( T \) in \( G \) with maximum degree at most \( k = 2 \). Since \( T \) spans all vertices in \( G \) and has \( |V| - 1 \) edges:
    \begin{itemize}
        \item The vertex \( v_0 \) must be connected to exactly one other vertex in \( T \) (because its degree is at most 2).
        \item Removing \( v_0 \) and its incident edge from \( T \) leaves a spanning tree of \( G' \) with maximum degree at most 2.
    \end{itemize}
    This remaining tree corresponds to a Hamiltonian path in \( G' \) because it connects all vertices in \( G' \) in a path-like manner (since every vertex in \( G' \) has degree at most 2).
\end{itemize}

Thus, we have a polynomial-time reduction from the Hamiltonian Path Problem to our problem, showing that our problem is NP-hard.

\miquestion[50]
In the \emph{$k$-means} problem, you are given a set of data points $D=\{\mathbf{x}_1,\ldots,\mathbf{x}_n\in\mathbb{R}^d\}$ and a positive integer $k$ as inputs, and you need to output $k$ ``centers'' $\mathbf{c}_1,\ldots,\mathbf{c}_k\in\mathbb{R}^d$ and a $k$-partition $(C_1,\ldots,C_k)$ of the data points $D$ such that the data points in $C_i$ is assigned to the center $\mathbf{c}_i$.
The objective is to minimize the following value
$$\sum_{i=1}^k\sum_{\mathbf{x}\in C_i}\|\mathbf{x}-\mathbf{c}_i\|^2$$
which is the sum of the squared distances from the data points to their assigned centers.

Prove that the following problem is NP-complete:
given a $k$-means instance $(D,k)$ and a non-negative value $\theta$, decide if there exists a solution $((\mathbf{c}_1,\ldots,\mathbf{c}_k),(C_1,\ldots,C_k))$ that makes the objective value at most $\theta$.

Solution:

To prove that the problem is NP-complete, we need to show that it is in NP and that it is NP-hard by reducing another NP-complete problem to it. 

Step 1: Proving NP-membership

Given a solution $((\mathbf{c}_1,\ldots,\mathbf{c}_k),(C_1,\ldots,C_k))$, we can verify in polynomial time whether the objective value is at most $\theta$. This is because computing the squared distance of each data point to its assigned center and summing these distances has a polynomial time complexity.

**Step 2: Reduction from a known NP-complete problem, such as Subset Sum**

We can reduce the Subset Sum problem to the $k$-means problem as follows:

Given an instance of Subset Sum with a set of integers $S=\{a_1, a_2, ..., a_n\}$ and a target value $T$, we construct a $k$-means instance as follows:

- Let $D = \{a_1, a_2, ..., a_n\}$ be the set of data points.
- Let $k = 2$.
- Set $\theta = T^2$.

Now, we claim that there exists a solution to the $k$-means instance with objective value at most $\theta$ if and only if there exists a subset of $S$ that sums up to $T$.

If there exists a subset of $S$ that sums up to $T$:

Let $S'$ be such a subset. Define two centers: $\mathbf{c}_1 = \frac{1}{2}(T, 0, ..., 0)$ and $\mathbf{c}_2 = \frac{1}{2}(-T, 0, ..., 0)$. Assign each data point $a_i \in S'$ to $\mathbf{c}_1$ and each data point $a_i \notin S'$ to $\mathbf{c}_2$. The objective value of this solution is:
\[
\sum_{a_i \in S'} \|a_i - \mathbf{c}_1\|^2 + \sum_{a_i \notin S'} \|a_i - \mathbf{c}_2\|^2 = \sum_{a_i \in S'} \left(\frac{T - a_i}{2}\right)^2 + \sum_{a_i \notin S'} \left(\frac{-T - a_i}{2}\right)^2
\]
\[= \frac{1}{4} \left(\sum_{a_i \in S'} (T - a_i)^2 + \sum_{a_i \notin S'} (-T - a_i)^2\right) = \frac{1}{4} \left(\sum_{a_i \in S'} (T^2 - 2Ta_i + a_i^2) + \sum_{a_i \notin S'} (T^2 + 2Ta_i + a_i^2)\right)\]
\[= \frac{1}{4} \left(\sum_{a_i \in S'} (T^2 - 2Ta_i + a_i^2) + \sum_{a_i \in S'} (T^2 + 2Ta_i + a_i^2)\right) = \frac{1}{4} \left(|S'|T^2 + |S'|(T^2)\right) = \frac{1}{4}(2|S'|T^2)\]
\[= \frac{1}{2}|S'|T^2 = \frac{1}{2}T^2 = \theta.\]

If there exists a solution to the $k$-means instance with objective value at most $\theta$:

This implies that there are two centers $\mathbf{c}_1$ and $\mathbf{c}_2$ such that the sum of squared distances of data points to their assigned centers is at most $\theta$. Since $k = 2$, there are two clusters. Let $S'$ be the set of data points assigned to $\mathbf{c}_1$. Then the sum of squared distances of data points in $S'$ to $\mathbf{c}_1$ is $\sum_{a_i \in S'} \|a_i - \mathbf{c}_1\|^2 = |S'| \cdot 0 = 0$. The sum of squared distances of data points in $D \setminus S'$ to $\mathbf{c}_2$ is $\sum_{a_i \notin S'} \|a_i - \mathbf{c}_2\|^2 = \sum_{a_i \notin S'} (a_i + T)^2$. This implies that for each $a_i \notin S'$, we have $(a_i + T)^2 \leq \theta$. Since each $(a_i + T)^2$ is non-negative, this implies that $a_i \leq \sqrt{\theta}$.

Hence, if there exists a solution to the $k$-means instance with objective value at most $\theta$, then there exists a subset of $S$ such that the sum of its elements is at most $\sqrt{\theta}$.

Therefore, the $k$-means problem with a non-negative value $\theta$ is NP-complete. 
  
  
\miquestion
How long does it take you to finish the assignment (including thinking and discussion)?
Give a score (1,2,3,4,5) to the difficulty.
Do you have any collaborators?
Please write down their names here.

The diffitulty score is 5. I used ChatGPT to help me translate text, layout, and write formulas during the homework completion process.

\end{questions}


\end{document}
