%Example of use of oxmathproblems latex class for problem sheets
\documentclass{oxmathproblems}

%(un)comment this line to enable/disable output of any solutions in the file
%\printanswers

%define the page header/title info
\course{Algorithm Design and Analysis}
\sheettitle{Assignment 5 \\ Deadline: Jun 10, 2024} %can leave out if no title per sheet


\begin{document}

\begin{questions}
\miquestion[25] \textbf{[Common System of Distinct Representatives]}
Given a ground set $U=\{1,\ldots,n\}$ and a collection of $k$ subsets $\mathcal{A}=\{A_1,\ldots,A_k\}$, a \emph{system of distinct representatives} of $\mathcal{A}$ is a ``representative'' collection $T$ of distinct elements from the sets in $\mathcal{A}$.
Specifically, we have $|T|=k$, and the $k$ \emph{distinct} elements in $T$ can be ordered as $u_1,\ldots,u_k$ such that $u_i\in A_i$ for each $i=1,\ldots,k$.
For example, $\{A_1=\{2,8\},A_2=\{8\},A_3=\{4,5\},A_4=\{2,4,8\}\}$ has a system of distinct representatives $\{2,4,5,8\}$ where $2\in A_1,4\in A_4,5\in A_3,8\in A_2$, while $\{A_1=\{2,8\},A_2=\{8\},A_3=\{4,8\},A_4=\{2,4,8\}\}$ does not have a system of distinct representatives.
\begin{parts}
\part[10] Design a polynomial time algorithm to decide if $\mathcal{A}$ has a system of distinct representatives.
\part[15] Given a ground set $U=\{1,\ldots,n\}$ and two collections of $k$ subsets $\mathcal{A}=\{A_1,\ldots,A_k\}$ and $\mathcal{B}=\{B_1,\ldots,B_k\}$, a \emph{common system of distinct representatives} is a collection $T$ of $k$ elements that is a system of distinct representatives of both $\mathcal{A}$ and $\mathcal{B}$. Design a polynomial time algorithm to decide if $\mathcal{A}$ and $\mathcal{B}$ have a common system of distinct representatives.
\end{parts}
For each part, prove the correctness of your algorithm, and analyze its time complexity.

Solutions:

\begin{parts}
    \part This problem is similar to the maximum matching problem. We can construct a bipartite graph \(G\), where for each set \(A_u\) in \(\mathcal{A} = \{A_1, \ldots, A_k\}\), they are considered as the left-side vertex set of \(G\), and the elements in \(U\) are the right-side vertex set. There exists an edge \((u, v)\) in \(G\) if and only if \(v \in A_u\). Then, construct a source vertex \(s\) that connects to all the left-side vertices and a sink vertex \(t\) that connects to all the right-side vertices. If the maximum flow in this directed graph is \(k\), then the corresponding set of edges is a solution to the problem. Due to the correctness of the solution to the maximum matching problem, the above solution is correct. Since there is a one-to-one correspondence between the left and right vertices, it is easy to determine that such a solution is valid. Its time complexity is the same as the time complexity of the maximum flow problem, which is \(O(VE^2)\). In this problem, \(|V| = n + k + 2\) and \(|E| \leq n + k + nk\), so an upper bound for the time complexity is \(O(nk(n + k))\).

    \part To solve this problem, we can use the Hungarian algorithm. First, we transform the problem into a maximum matching problem in a bipartite graph. We map each subset \(A_i\) in the collection \(\mathcal{A}\) to the left vertex set and each subset \(B_i\) in the collection \(\mathcal{B}\) to the right vertex set. Then, we add edges in this bipartite graph from each left vertex to each right vertex if the corresponding element exists in both subsets, indicating their commonality. Next, we apply the Hungarian algorithm to find the maximum matching in this bipartite graph. If the size of the maximum matching is equal to \(k\), then there exists a common system of distinct representatives; otherwise, it does not.

\textbf{Correctness Analysis:}
\begin{itemize}
    \item By construction, each matching in the bipartite graph represents a potential system of distinct representatives common to both collections \(\mathcal{A}\) and \(\mathcal{B}\).
    \item The Hungarian algorithm guarantees that it finds the maximum matching in the bipartite graph, ensuring that if there is a common system of distinct representatives, it will be found.
\end{itemize}

\textbf{Time Complexity Analysis:}
\begin{itemize}
    \item Constructing the bipartite graph takes \(O(kn)\) time, where \(n\) is the size of the ground set \(U\).
    \item Applying the Hungarian algorithm takes \(O(n^2k)\) time.
    \item Thus, the overall time complexity of the algorithm is \(O(n^2k)\).
\end{itemize}

\end{parts}

\miquestion[25] \textbf{[Hall Marriage Theorem]}
Given a bipartite graph $G=(A,B,E)$ with $|A|=|B|=n$, \emph{Hall Marriage Theorem} states that $G$ contains a perfect matching (i.e., a matching with size $n$) if and only if $|N(S)|\geq |S|$ for any $S\subseteq A$, where $N(S)=\{b\in B\mid \exists a\in S\subseteq A: (a,b)\in E\}$ is the set of all the neighbors of the vertices in $S$.
In this question, we are going to prove Hall Marriage Theorem.
\begin{parts}
\part[5] Prove the necessity part: if $G$ contains a perfect matching, then  $|N(S)|\geq |S|$ for any $S\subseteq A$.
\part[10] Construct a directed weighted graph with vertex set $V=A\cup B\cup\{s\}\cup\{t\}$. The edge set is defined as follows. For each $a\in A$, construct a directed edge $(s,a)$ with weight $1$; for each $a\in A$ and $b\in B$, construct a directed edge $(a,b)$ with weight $\infty$ if $(a,b)$ is an edge in the original graph $G$; for each $b\in B$, construct a directed edge $(b,t)$ with weight $1$.
Prove that, if $|N(S)|\geq |S|$ holds for all $S\subseteq A$, the minimum $s$-$t$ cut has value $n$.
\part[10] Prove the sufficiency part: if $|N(S)|\geq |S|$ holds for all $S\subseteq A$, then $G$ contains a perfect matching.
\end{parts}

Solutions:

\begin{parts}
    \part Suppose $G$ contains a perfect matching. Let $M$ denote this perfect matching. Now, consider any subset $S \subseteq A$. Since $M$ is a perfect matching, each vertex in $S$ must be matched to a distinct vertex in $B$. Therefore, the set $N(S)$ contains at least $|S|$ distinct vertices from $B$, implying $|N(S)| \geq |S|$. Hence, the necessity part is proved.

    \part We construct a directed weighted graph with vertex set $V=A\cup B\cup\{s\}\cup\{t\}$. The edge set is defined as follows:
    
    \begin{itemize}
        \item For each $a \in A$, we add a directed edge $(s,a)$ with weight $1$.
        \item For each $a \in A$ and $b \in B$, if $(a,b)$ is an edge in the original graph $G$, we add a directed edge $(a,b)$ with weight $\infty$.
        \item For each $b \in B$, we add a directed edge $(b,t)$ with weight $1$.
    \end{itemize}
    
    We aim to prove that if $|N(S)| \geq |S|$ holds for all $S \subseteq A$, then the minimum $s$-$t$ cut has value $n$.
    
    Let $f$ be any $s$-$t$ flow in the constructed graph. We define $S_f$ as the set of vertices reachable from $s$ in the residual graph of $f$. By the max-flow min-cut theorem, the value of the minimum $s$-$t$ cut is equal to the value of the maximum flow, which is the size of $S_f$. Since $S_f$ includes all vertices in $A$ and possibly some vertices in $B$, the size of $S_f$ is at least $|A|=n$. Additionally, since each edge $(s,a)$ has capacity $1$, for each $a \in A$, at most one edge $(s,a)$ can be part of the minimum $s$-$t$ cut. Therefore, the size of the minimum $s$-$t$ cut is $n$.


    \part Suppose $|N(S)| \geq |S|$ holds for all $S \subseteq A$. We aim to show that there exists a perfect matching in $G$.
    
    We prove this by contradiction. Assume that there is no perfect matching in $G$. By Hall's Marriage Theorem, this implies that there exists a subset $S \subseteq A$ such that $|N(S)| < |S|$. However, this contradicts our assumption that $|N(S)| \geq |S|$ holds for all $S \subseteq A$. Therefore, our assumption that there is no perfect matching is false, implying the existence of a perfect matching in $G$. Thus, the sufficiency part is proved.

\end{parts}

\miquestion[25]
For the linear program
\begin{align*}
\text{maximize }& x_1-2x_3 \\
\text{subject to }&x_1-x_2\leq1\\
&2x_2-x_3\leq1\\
&x_1,x_2,x_3\geq0
\end{align*}
prove that the solution $(x_1,x_2,x_3)=(3/2,1/2,0)$ is optimal.

Solution:

We have:

$x_1 \leq 1+x_2$

$-2x_3 \leq 2-4x_2$

Add it, we have:

$x_1-2x_3 \leq 3-3x_2$

To maximize $x_1-2x_3$, we should minimize $3-3x_2$, then maximize $x_2$. For $2x_2-x_3\leq1$, We can know that the maximum value of $x_2$ is $1/2$. Hence, the solution is optimal.

\miquestion[25]\textbf{[K\"{o}nig-Egerv\'{a}ry Theorem]}
In the class, we have seen that the maximum matching problem can be formulated by the following linear program
\begin{align*}
\text{maximize }& \sum_{e\in E}x_e \\
\text{subject to }&\sum_{e:e=(u,v)}x_e\leq 1  \tag{$\forall v\in V$}\\
&x_e\geq0\tag{$\forall e\in E$}
\end{align*}
and the minimum vertex cover problem can be formulated by the following linear program
\begin{align*}
\text{minimize }& \sum_{u\in V}x_u \\
\text{subject to }&x_u+x_v\geq 1  \tag{$\forall (u,v)\in E$}\\
&x_u\geq0\tag{$\forall u\in V$}
\end{align*}
We have also mentioned that the second linear program is the dual program of the first (try to verify it on your own).
\begin{parts}
\part[20] Prove that both linear programs have integral optimal solutions if the graph is bipartite.
\part[5] Using the result in the first part, prove K\"{o}nig-Egerv\'{a}ry Theorem, which states that the size of the maximum matching equals to the size of the minimum vertex cover in a bipartite graph.
\end{parts}

Solutions:

\begin{parts}

    \part To prove that both linear programs have integral optimal solutions if the graph is bipartite, let's first consider the maximum matching problem. In a bipartite graph, we can represent the edges as a set of disjoint pairs $(u,v)$, where $u$ belongs to one partition and $v$ belongs to the other.
    
    For each vertex $v$ in the graph, the constraint $\sum_{e
    =(u,v)}x_e \leq 1$ ensures that each vertex is incident to at most one edge in the matching. Since the graph is bipartite, each vertex has at most one incident edge in the matching. Therefore, the solution to this linear program corresponds to a maximum matching, and since each variable $x_e$ represents whether edge $e$ is included in the matching (with $x_e = 1$) or not (with $x_e = 0$), the solution will be integral.
    
    Now, let's consider the minimum vertex cover problem. In a bipartite graph, the minimum vertex cover can be obtained by selecting all vertices not covered by the maximum matching. Since the maximum matching covers all vertices incident to its edges, the remaining vertices form a vertex cover.
    
    The constraint $x_u + x_v \geq 1$ for each edge $(u,v)$ ensures that at least one of the vertices incident to an edge is included in the vertex cover. Since the graph is bipartite, there are no edges between vertices in the same partition, and each edge connects vertices from different partitions. Therefore, selecting any one of the vertices incident to each edge ensures that the minimum vertex cover covers all edges, and thus, it covers all vertices not included in the maximum matching.
    
    Since both linear programs have integral optimal solutions in a bipartite graph, the maximum matching has the same size as the minimum vertex cover.

    \part Using the result from the first part, we can now prove König-Egerváry Theorem. In a bipartite graph, the size of the maximum matching equals the size of the minimum vertex cover.

    By the definition of the maximum matching and minimum vertex cover, and the result proved in the first part, we know that both the maximum matching and the minimum vertex cover have integral optimal solutions in a bipartite graph. Therefore, the number of edges in the maximum matching, which corresponds to the objective value of the maximum matching linear program, is equal to the number of vertices in the minimum vertex cover, which corresponds to the objective value of the minimum vertex cover linear program.
    
    Hence, the size of the maximum matching equals the size of the minimum vertex cover in a bipartite graph, which is the König-Egerváry Theorem.

\end{parts}
\miquestion
How long does it take you to finish the assignment (including thinking and discussion)?
Give a score (1,2,3,4,5) to the difficulty.
Do you have any collaborators?
Please write down their names here.
 
I think this assignment is very difficult, and there is still a part of the reason that I am not familiar with English, which makes it difficult for me to understand the meaning of the problem itself. If the exam is also of this difficulty, it will be bad. The difficulty score is 5.
\end{questions}


\end{document}
