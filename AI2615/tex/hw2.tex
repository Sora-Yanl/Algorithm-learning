%Example of use of oxmathproblems latex class for problem sheets
\documentclass{oxmathproblems}

%(un)comment this line to enable/disable output of any solutions in the file
%\printanswers

%define the page header/title info
\course{Algorithm Design and Analysis}
\sheettitle{Assignment 2 \\ Deadline: April 15, 2024} %can leave out if no title per sheet


\begin{document}

\begin{questions}
\miquestion[25]
Let $G=(V,E)$ be an undirected connected graph.
\begin{parts}
    \part Prove that you can sort the vertices $(v_1,\ldots,v_n)$ such that, for any $k\in\{1,\ldots,n\}$, the subgraph induced by the vertex set $V_k=\{v_1,\ldots,v_k\}$ is connected.
    \part Does the claim hold for directed and strongly connected graphs? That is, given a directed graph $G=(V,E)$ that is strongly connected, can we sort the vertices $(v_1,\ldots,v_n)$ such that the subgraph induced by $V_k=\{v_1,\ldots,v_k\}$ is strongly connected for any $k\in\{1,\ldots,n\}$? If so, prove it; if not, provide a counterexample.
\end{parts}

Solutions:
\begin{parts}
    \part We can refer to the idea of "breadth-first search"(BFS).First initialize a queue, choose a vertex and push it into the queue, then repeat the following steps until the queue is empty: pop the first vertex into the queue and record it, then push all vertexs(which is not recorded) connected to it in the queue. Because $G$ is a connected graph, we can definitely traverse all vertexs. Similarly, due to the properties of connected graphs, we can ensure that the dequeue order is a answer that satisfies the problem.
    \part This sorting method do not exists. Consider a counterexample: $V = \{v_1, v_2, v_3\}, E = \{v_1->v_2, v_2->v_3, v_3->v_1\}$, which is a directed and strongly connected graph. for $k=2$, we cannot find two vertexs and edges, which can form a strongly connected graph.
\end{parts}

\miquestion[25]
Let $G=(V,E)$ be a directed graph and $s,t$ be two vertices.
An $(s,t)$-\emph{Hamiltonian path} is a path from $s$ to $t$ that visits each vertex exactly once.
Design a polynomial time algorithm that, given a \emph{directed acyclic graph} $G=(V,E)$ and two vertices $s,t$ as inputs, decides if $G$ contains an $(s,t)$-Hamiltonian path.
Prove the correctness of your algorithm, and analyze its running time.

\noindent\textbf{Remark: }We believe this problem does not admit polynomial-time algorithms on general graphs (that is not necessarily acyclic). In the last chapter of this course, we will prove that this problem is NP-complete.

Solution:

First, we need to check if there exists a path from $s$ to $t$. If such a path does not exist, then $G$ definitely does not contain an $(s,t)$-Hamiltonian path.

If there exists a path from $s$ to $t$, we can use a modified version of topological sorting and DFS to check for the existence of an $(s,t)$-Hamiltonian path. We can perform a depth-first search starting from $s$, checking during the search process if all vertices are visited exactly once.

Time Complexity Analysis:

Checking the existence of a path from $s$ to $t$: $O(V+E)$.

modified topological sorting algorithm: $O(V+E)$.

If there is no path from $s$ to $t$, then $G$ definitely does not contain an $(s,t)$-Hamiltonian path, and the algorithm correctly identifies this.

If there exists a path from $s$ to $t$, we can find a path from $s$ to $t$ using depth-first search while ensuring all vertices are visited exactly once. Such a path is an $(s,t)$-Hamiltonian path. Therefore, the algorithm is correct.

\miquestion[25] 
We have seen in the class that Dijkstra algorithm fails if the graph contains negatively-weighted edges.
Consider the following variant of Dijkstra algorithm.
Given a directed weighted graph $G=(V,E,w)
$ where $w(u,v)$ may be negative, find an integer $W$ such that $w(u,v)+W>0$ for each edge $(u,v)\in E$, and define the new weight of $(u,v)$ as $w'(u,v)=w(u,v)+W$.
Now, $G'=(V,E,w')$ is a positively weighted graph where the weight of each edge is increased by $W$.
Implement Dijkstra algorithm on $G'=(V,E,w')$, and a shortest path from $s$ to every vertex $u$ in $G'$ is also a shortest path in $G$.

\begin{parts}
\part[10] Does this algorithm work for directed acyclic graphs? If so, prove it; if not, provide a counterexample.
\part[15] A \emph{directed grid} is a directed weighted graphs $G=(V,E,w)$ where the set of vertices is given by $V=\{v_{ij}\mid i=1,\ldots,m;j=1,\ldots,n\}$ ($m,n\in\mathbb{Z}^+$ are two parameters) and the set of directed edges is given by $$E=\left(\bigcup_{i=1,\ldots,m-1;j=1,\ldots,n}\left\{(v_{ij},v_{(i+1)j})\right\}\right)\cup\left(\bigcup_{i=1,\ldots,m;j=1,\ldots,n-1}\left\{(v_{ij},v_{i(j+1)})\right\}\right).$$
That is, the vertices form a $m\times n$ grid. There is a \emph{directed} edge \emph{from} every vertex \emph{to} the vertex ``right above'' it, and there is a \emph{directed} edge \emph{from} every vertex \emph{to} the vertex ``to the right of'' it (unless the vertex is ``on the boundary'').
The weight of each edge is specified as input and can be negative.

Does the algorithm work for directed grids? If so, prove it; if not, provide a counterexample.
\end{parts}

Solutions:
\begin{parts}
    \part This algorithm does not work. Consider a counterexample: $V = \{v_1, v_2, v_3, v_4\}, E = \{v_1->v_3: 5, v_1->v_2: 1, v_2->v_3: 2, v_1->v_4: -4\}$, Let $W=5$, then the weights of all edges are positive in $G'$. Now the shortest past from $v_1$ to $v_3$ in $G$ is $v_1->v_2->v_3$, but $v_1->v_3$ in $G'$. So the shortest path from $s$ to $t$ in $G'$ may be not a shortest path in $G$.
    \part This algorithm works. because the path can only be "up" or "right", refering to mathematical induction method, we can prove its correctness when $m, n=1, 2$. If it holds when $m, n=a-1, b-1$, for $m, n=a, b-1$, it becomes a subproblem of scale 1 again.
\end{parts}

\miquestion [25]
  Given a directed graph $G=(V,E)$ where each vertex can be viewed as a port. Consider that you are a salesman, and you plan to travel the graph. Whenever you reach a port $v$, it earns you a profit of $p_v$ dollars, and it cost you $c_{uv}$ if you travel from $u$ to $v$. For any directed cycle in the graph, we can define a profit-to-cost ratio to be
  $$
    r(C) = \frac{\sum_{(u,v)\in C}p_v}{\sum_{(u,v)\in C}c_{uv}}.
  $$ 
  As a salesman, you want to design an algorithm to find the best cycle to travel with the largest profit-to-cost ratio. Let $r^*$ be the maximum profit-to-cost ratio in the graph. 
  \begin{parts}
    \part If we guess a ratio $r$, can we determine whether $r^*>r$ or $r^*<r$ efficiently? Design an algorithm to determine this. Prove the correctness of your algorithm, and analyze its running time.
    \part Based on the guessing approach, given a desired accuracy $\epsilon>0$, design an efficient algorithm to output a good-enough cycle, where $r(C) \geq r^* - \epsilon$. Justify the correctness and analyze the running time in terms of $|V|$, $\epsilon$, and $R=\max_{(u,v)\in E}(p_u/c_{uv})$.
  \end{parts}
  (Hint: You may want to construct another graph with appropriate edge weights, and use an appropriate shortest path algorithm.)

Solutions:
\begin{parts}
    \part Observe that if there exists a profit-to-cost ratio $r^*$ greater than $r$. Thus, we can design an algorithm to check if there exists a cycle whose profit-to-cost ratio is greater than $r^*$. The algorithm proceeds as follows:
    
    Use the Bellman-Ford or Floyd-Warshall algorithm to find all shortest paths in the graph.
    
    Perform a topological sort on all vertices.
    
    Starting from the source vertex, compute the maximum profit-to-cost ratio for each vertex in topological order. For each vertex $v$, update its maximum profit-to-cost ratio to $\max(\text{current ratio}, \text{ratio of upstream vertex} + \text{adjustment value of the edge})$.
    
    If a profit-to-cost ratio greater than $r$ is found, return True; otherwise, return False.
    
    The time complexity of the algorithm is $O(|V|\cdot|E|)$.
    \part For a given precision $\epsilon$, we can employ binary search to find the maximum $r$ such that there exists a cycle with a profit-to-cost ratio greater than or equal to $r-\epsilon$. Then, we can use the above algorithm to check if such a cycle exists.

    The algorithm proceeds as follows:
    
    Initialize the left and right boundaries of the binary search to 0 and $R$ respectively, where $R=\max_{(u,v)\in E}(p_u/c_{uv})$.
    In each iteration, compute the current guess ratio $r$ and use the above algorithm to check if there exists a cycle with a profit-to-cost ratio greater than or equal to $r-\epsilon$.
    If such a cycle exists, update the left boundary to the current guess ratio; otherwise, update the right boundary to the current guess ratio.
    Repeat steps 2 and 3 until the difference between the left and right boundaries is less than $\epsilon$.
    Return the left boundary as the final guessed ratio.
    The correctness of the algorithm is guaranteed by the properties of binary search. The time complexity of the algorithm is $O(|V|\cdot|E| \log R / \epsilon)$.
\end{parts}
  
\miquestion
How long does it take you to finish the assignment (including thinking and discussion)?
Give a score (1,2,3,4,5) to the difficulty.
Do you have any collaborators?
Please write down their names here.
 
I spent about three hours completing this assignment.

I think the difficulty score is 5. The first question is relatively easy, while the following questions are more difficult, especially 3 and 4.

I used ChatGPT when answering, but due to the complexity of the problem, it did not provide me with much help in solving it. I think the answers of AI are not always correct, and questions that rely on searching for information cannot be called easy. Also, my English proficiency also influenced my answers to a certain extent

\end{questions}


\end{document}
